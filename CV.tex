\documentclass{article}

\usepackage{xcolor}
\usepackage{hyperref}
\definecolor{urlblue}{HTML}{0000CD}
\definecolor{darkgrey}{HTML}{191919}

\hypersetup{
    colorlinks=true,
    linkcolor=urlblue,
    urlcolor=darkgrey,
}

\usepackage{enumitem}
\renewcommand{\labelitemi}{$\bullet$}
\setlist{leftmargin=2em}

\renewcommand\familydefault\sfdefault

% Contact details
\newcommand{\contact}[1]{\normalsize{#1}}
\newcommand{\contactdiv} {\hspace*{0.8em}}
\newcommand{\email}[2]{\href{mailto:#1@#2}{\underline{#1{\small\fontfamily{phv}\selectfont@}#2}}}
\newcommand{\linkedin}{\href{https://www.linkedin.com/in/joshua-walker-080714238/}{\underline{LinkedIn}}}
\newcommand{\portfolio}{\href{https://josh-ja-walker.github.io/portfolio/}{\underline{Portfolio}}}

% Layout helpers    
\newcommand{\horizrule}{{\vspace{2pt}}{\hrule height 0.5pt}}
\newcommand{\dates}[1]{\hfill\textit{(#1)}}

% A Level results
\newcommand{\alevel}[2]{#1 - \textit{\textbf#2}}

\newcommand{\tab}{\hspace{1em}}
% Project formatting
\newcommand{\project}[4]{
    \subsection*{\textbf{#1} - \textit{#3} \dates{#2}}\hfill
    \begin{minipage}{\dimexpr\textwidth - 1em}
        #4
    \end{minipage}
}

% Layout header formatting
\usepackage{titlesec}
\titleformat{\section}{\normalfont\large}{\thesection}{}{}[\horizrule]
\titlespacing*{\section}{0pt}{1em}{0.65em}

\titleformat{\subsection}{\normalfont}{\thesection}{}{}
\titlespacing*{\subsection}{0pt}{0.5em}{0.45em}

% Column handling
\usepackage{multicol}
\setlength{\columnsep}{2em}

% Turn off numbering
\pagenumbering{gobble}

% Set margin
\usepackage[a4paper, total={7.5in, 10.5in}]{geometry}


\begin{document}

% Header
\begin{multicols}{2}

    % Name and links
    \begin{flushleft}
        {\LARGE\textbf{Joshua Walker}}\\
        \vspace{0.2em}
        \portfolio\contactdiv\linkedin
    \end{flushleft}
    
    \columnbreak
    
    % Contact details
    \begin{flushright}
        \contact{Acton, London, W3 8LU}\\
        \vspace{0.2em}
        \contact{\email{j.walker1005}{outlook.com}}
        \contactdiv
        \contact{+44 7572224173}
    \end{flushright}
    
\end{multicols}

\vspace{0.15em}

% Profile
\begin{minipage}{\dimexpr\textwidth - 1em}
    MEng Computing student in my \textbf{penultimate year} at Imperial College London looking for a 
    \textbf{6-month placement} from April - September. I have an interest in \textbf{games and software development}, 
    explored through game jams and university projects which helped me build foundational teamworking skills 
    and further my interest in design and implementation.
\end{minipage}

\vspace{0.15em}

\section*{Education}

% University
\subsection*{\textbf{Imperial College London} \dates{2022 - current}}

{\tab}MEng Computing course, currently in 3rd year. Achieved \textbf{first class honours} in 2nd year


% A Levels
\subsection*{\textbf{Shrewsbury Sixth-Form College} \dates{2020 - 2022}}

{\tab}A Levels: 
    \alevel{Computer Science}{A*},
    \alevel{Physics}{A*},
    \alevel{Maths}{A*},
    \alevel{Further Maths}{A}



\section*{Work Experience and Volunteering}

\subsection*{\textbf{IC Hack} \dates{2023, 2024}}
{\tab}Europe's largest student-run hackathon, hosted on South Kensington Campus

\vspace{-0.75em} 
\begin{itemize}[noitemsep]
    \item Volunteered to help secure the premises and cater for the hundreds of participants and industry representatives
    \item Participated to learn more about different sectors of the industry through the workshops, 
    while experiencing the challenge of the hackathon.
\end{itemize}
\vspace{-0.75em} 


\subsection*{\textbf{In-Comm} (formerly Marches Centre of Manufacturing Technology) \dates{2019}}
{\tab}Completed a week's work experience among a group of automotive apprentices, studying engines in a practical environment to 
    better understand how software engineering techniques can be applied to different engineering disciplines.



\section*{Projects}

\project{Lettuce Eat \normalfont{(Group Project)}}{2024}{Flutter, Firebase, Dart}{
    Studied \textbf{Human-Centered Design techniques} with the Royal College of Art in order to build a student recipe-based social media Android app.
    Applied paired-programming and iterative techniques to build the frontend, whilst learning about the difficulties of fullstack development
    from teammates who worked on the backend.
}

\project{WACC Compiler \normalfont{(Group Project)}}{2024}{Rust}{
    Built an optimising compiler from scratch in \textbf{Rust}, using the CLAP, Chumsky and Ariadne crates to provide a command-line interface
    and clear error reporting. Designed using internal intermediate representation to allow for code generation into both
    \textbf{Arm32} and \textbf{Intel x86-64} assembly code.
}

\project{Sorts TUI}{2024}{Rust}{
    Developed an in-terminal utility for visualising sorting algorithms - built entirely in Rust using the Ratatui and CLAP crates.
}

\project{Pintos \normalfont{(Group Project)}}{2023}{C}{
    Extended a simple \textbf{operating system framework} into a complex and developed system with features such as a priority-based thread scheduler, 
    synchronisation primitives, system calls and virtual memory. Explored foundational concepts of OS architectures and delved into the
    pre-existing codebase to understand their practical implementations. Managed workload between group members and wrote detailed design reports. 
}

\project{Weather Wallpaper}{2022 - 2024}{Rust}{
    Built a \textbf{command-line wallpaper engine} that dynamically changes the user's desktop wallpaper to reflect the local weather. 
    Learnt how to access online API data and used the Dialoguer crate to provide a clean terminal menu interface for selecting weather tags for 
    user-provided wallpaper images.
}

\project{Origame}{2022 - 2024}{Unity, C\#}{
    Designed and developed a \textbf{2D puzzle platformer game} for A-Level coursework - attained an \textbf{A/A*} grade
    and has been expanded upon after submission. Learnt problem-solving and bug-fixing techniques during the design of the complex central mechanic: 
    'folding' paper platforms in order to progress, providing the illusion using vector mathematics and sprite redrawing.
    Practiced \textbf{level and puzzle design}, and learnt how to limit a design appetite to maintain a cohesive vision.  
    Completed a comprehensive design document explaining game's inspirations and design methodology.
}

\project{Painball}{2021}{Unity, C\#}{
    Pinball game developed in \textbf{7 days} for the 2021 Ludwig Jam. The theme of the jam was Foddian: a game
    which emulates the frustration of popular Bennett Foddy games.
    I learnt the importance of \textbf{user testing and feedback} - by watching people play, I could calibrate the difficulty
    to ensure the game remained fun while still meeting the design brief. 
}
    

\section*{Skills and Interests}

\textbf{Languages and Technologies:}{\tab}
    C, C++, C\#, Rust, Java, Kotlin, Haskell, React, JS, HTML, CSS, %Flutter, 
    Git, Unity, Latex

% \noindent\textbf{Game Development:} I have competed in game jams, developing games in Unity in anywhere from 48 hours to 14 days.
% To learn more about the industry, I have studied game development post-mortem blogs, and attended the Games Careers Week.\\

\noindent\textbf{Archery:}{\tab}former member of Audco Archers, an archery club based at Archery GB's training facilities;
training with Olympic and Paralympic archers was hugely inspiring.

\noindent\textbf{Guitar:}{\tab}self-taught guitarist; I learnt acoustic guitar to keep busy during the Covid-19 lockdown
and have kept playing since.
\end{document}