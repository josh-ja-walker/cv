\documentclass{article}

\usepackage{xcolor}
\usepackage{hyperref}
\definecolor{urlblue}{HTML}{0000CD}
\definecolor{darkgrey}{HTML}{191919}

\hypersetup{
    colorlinks=true,
    linkcolor=urlblue,
    urlcolor=darkgrey,
}

\usepackage{enumitem}
\renewcommand{\labelitemi}{$\bullet$}

\renewcommand\familydefault\sfdefault

% Contact details
\newcommand{\contact}[1]{\normalsize{#1}}
\newcommand{\contactdiv}{\hspace*{0.8em}}
\newcommand{\email}[2]{\href{mailto:#1@#2}{\underline{#1{\small\fontfamily{phv}\selectfont@}#2}}}
\newcommand{\linkedin}{\href{https://www.linkedin.com/in/joshua-walker-080714238/}{\underline{LinkedIn}}}
\newcommand{\portfolio}{\href{https://josh-ja-walker.github.io/portfolio/}{\underline{Portfolio}}}

% Layout helpers    
\newcommand{\horizrule}{{\vspace{2pt}}{\hrule height 0.5pt}}
\newcommand{\dates}[1]{\hfill\textit{(#1)}}
\newcommand{\tab}{\hspace{1.5em}}
\newcommand{\indentedblock}[1]{
    \hfill
    \begin{minipage}{\dimexpr\textwidth - 1.95em}
        #1
    \end{minipage}
}

% A Level results
\newcommand{\alevel}[2]{#1 - \textit{#2}}

% Project formatting
\newcommand{\project}[4]{
    \subsection*{\textbf{#1} - \textit{#3} \dates{#2}}
    \begin{itemize}[topsep=0pt, itemsep=0.05em]
        #4
    \end{itemize}
}

% Layout header formatting
\usepackage{titlesec}
\titleformat{\section}{\normalfont\large}{\thesection}{1em}{}[\horizrule]

\titleformat{\subsection}{\normalfont}{\thesection}{1em}{}

% Column handling
\usepackage{multicol}
\setlength{\columnsep}{2em}

% Turn off numbering
\pagenumbering{gobble}

% Set margin
% \usepackage[a4paper, total={7.5in, 10.5in}]{geometry}
\usepackage[a4paper, total={6.75in, 10in}]{geometry}

\begin{document}

% Header
\begin{multicols}{2}

    % Name and links
    \begin{flushleft}
        {\LARGE\textbf{Joshua Walker}}\\
        \vspace{0.5em}
        \portfolio\contactdiv\linkedin
    \end{flushleft}
    
    \columnbreak
    
    % Contact details
    \begin{flushright}
        \contact{Acton, London, W3 8LU}\\
        \vspace{0.5em}
        \contact{\email{j.walker1005}{outlook.com}}
        \contactdiv
        \contact{+44 7572224173}
    \end{flushright}
    
\end{multicols}

\vspace{0.8em}

% Profile
{\noindent}MEng Computing student in my \textbf{penultimate year} at Imperial College London looking for a \textbf{6-month placement} from April - September. 
Interested in games and software development, which I fostered through university projects and solo development work 
that helped me build foundational teamworking and time-management skills. 

\vspace{0.15em}


\section*{Education}

% University
\subsection*{\textbf{Imperial College London} \dates{2022 - current}}

{\tab}MEng Computing course, currently in 3rd year. Achieved \textbf{first class honours} in 2nd year


% A Levels
\subsection*{\textbf{Shrewsbury Sixth-Form College} \dates{2020 - 2022}}

{\tab}A Levels: 
    \alevel{Computer Science}{A*},
    \alevel{Physics}{A*},
    \alevel{Maths}{A*},
    \alevel{Further Maths}{A}



\section*{Work Experience and Volunteering}

\subsection*{\textbf{IC Hack} \dates{2023, 2024}}
{\tab}Europe's largest student-run hackathon, hosted on South Kensington Campus

\begin{itemize}[topsep=0.5em]
    
    \item \textbf{Volunteered} throughout both days to help secure the premises and cater for the hundreds of participants and industry representatives. 
    \textbf{Collaborated} with fellow students to \textbf{organise} and deliver a smooth experience for all involved, helping to handle unexpected logistical problems such as late 
    food deliveries. I learnt how to \textbf{handle pressure} and de-escalate situations while serving food to customers, 
    managing queues and disgruntled participants.

    \item Participated to discover more about different sectors of the industry through \textbf{technical workshops}, 
    while experiencing the challenge of the hackathon and the strict \textbf{time-management skills} required.
    I spoke to sponsors and industry representatives to better understand the practical application of concepts studied in lectures, 
    and developed integral \textbf{networking skills}.

\end{itemize}


\subsection*{\textbf{In-Comm} (formerly Marches Centre of Manufacturing Technology) \dates{2019}}
\indentedblock{
    Completed a week's work experience among a group of automotive apprentices, \textbf{integrating seamlessly} into an established group.
    Picked up practical \textbf{problem-solving skills} deconstructing retired engine blocks, collaborating with the apprentices to 
    overcome issues within a short timeframe whilst learning techniques used in different engineering disciplines. 
}


\section*{Projects}

\project{Lettuce Eat \normalfont{(Group Project)}}{2024}{Flutter, Firebase, Dart}{
    \item Studied \textbf{Human-Centered Design techniques} with the Royal College of Art in order to build a student recipe-based social media Android app.
    Determined the needs of the target market through interviews and \textbf{data-gathering techniques} to ensure our product was commercially viable.
    
    \item For the final deadline, my teammates and I presented the application to lecturers and fellow students, \textbf{ably demonstrating}
    the value of our project. When technical difficulties hampered the presentation, I improvised, holding the attention of the audience
    while my teammates fixed the errors.
}


\project{Sorts TUI}{2024}{Rust}{
    \item Developed an in-terminal utility for visualising sorting algorithms, built as as an exercise in understanding algorithmic complexity.
    
    \item The stark contrast between developing this command-line interface and Lettuce Eat, an Android app, taught me the degree at which 
    a commercial product is informed by its target market and their requirements.
}

\project{Portfolio Website}{2024}{JavaScript, React, CSS}{
    \item Designed and developed my own portfolio website to showcase skills and experience gained through university and solo projects.

    \item Learnt React and JavaScript fundamentals and how to deploy a website using GitHub pages, while exploring \textbf{website design} and its impact on 
    user experience.
}

\project{WACC Compiler \normalfont{(Group Project)}}{2024}{Rust}{
    \item Built an optimising compiler from scratch in Rust, using the clap, Chumsky and Ariadne crates to provide a command-line interface
    and clear error reporting. Designed using internal intermediate representation to allow for code generation into both
    Arm32 and Intel x86-64 assembly code.

    \item Worked effectively in \textbf{time-constrained project} - contributed and discussed design ideas as a group and led the group to 
    amicable solutions when dissenting opinions were offered. 
}

\project{Pintos \normalfont{(Group Project)}}{2023}{C}{
    \item Extended a simple \textbf{operating system framework} into a complex and developed system with features such as a priority-based thread scheduler, 
    synchronisation primitives, system calls and virtual memory. Explored foundational concepts of OS architectures and delved into the
    pre-existing codebase to \textbf{understand and research} their practical implementations. 

    \item Managed workload effectively between group members and wrote detailed design reports to evidence our decisions. 
}

\project{Weather Wallpaper}{2022 - 2024}{Rust}{
    \item Built a \textbf{command-line wallpaper engine} that changes the user's desktop wallpaper to reflect local weather. 

    \item Learnt how to access and deserialise online API data; used the Dialoguer crate to provide a clean interface for selecting weather
    depicted in user-provided wallpapers.
}

\project{Origame}{2022 - 2024}{Unity, C\#}{
    \item Developed a 2D puzzle platformer game for A-Level coursework - attained an \textbf{A/A*} grade
    and expanded further after submission. 

    \item Learnt crucial \textbf{problem-solving and debugging} skills when building the central mechanic: 
    providing the illusion of folding paper platforms using vector mathematics and sprite redrawing.
    Tackled level and puzzle design while learning to restrict scope creep.

    \item Completed a \textbf{comprehensive design document} explaining game's inspirations and design methodology.
}

\project{Painball}{2021}{Unity, C\#}{
    \item Pinball game developed in \textbf{7 days} for the 2021 Ludwig Jam. The theme of the jam was Foddian; 
    participants were tasked with creating a game which emulates the frustration of popular Bennett Foddy games.

    \item I learnt the importance of \textbf{user testing and feedback}: by watching people play, I could calibrate the difficulty
    to ensure the game remained fun whilst adhering to the design brief. 
}
    

\section*{Skills and Interests}

\subsection*{\textbf{Languages and Technologies}: C, C++, C\#, Rust, Java, Haskell, React, JS, HTML, CSS, Flutter, Git, Unity, Latex}
\vspace{0.5em}

\subsection*{\textbf{Archery}}
\vspace{-0.5em}
\indentedblock{
    I am a former member of Audco Archers, an archery club based at Archery GB's training facilities.
    Training with Olympic and Paralympic archers was hugely inspiring; among talented archers of all ages, 
    I learnt the value of \textbf{friendly competition}, something that pushes me to produce exceptional work alongside 
    my fellow coursemates.
}

\subsection*{\textbf{Guitar}}
\vspace{-0.5em}
\indentedblock{
    I am a \textbf{self-taught} guitarist; I learnt acoustic guitar to stay productive during the Covid-19 lockdown and have kept playing ever since. 
    Learning an entirely new skill was a daunting task, but through gradual improvement, I built confidence in my ability to commit 
    to rigorous self-study.
}

\end{document}